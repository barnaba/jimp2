%\documentclass[a4paper]{scrartcl}
\documentclass[a4paper]{article}
\usepackage{setspace}
\usepackage{url}
\usepackage{mdwlist}
\usepackage{polski}
\usepackage[utf8x]{inputenc}
\usepackage{color}
\usepackage{mathtools}
\usepackage{graphicx}
\usepackage[unicode=true]{hyperref}
\usepackage{multirow}
\usepackage[table]{xcolor}
\usepackage{subfig}
\usepackage{listings}
\definecolor{dkgreen}{rgb}{0.2,0.8,0.2}
\definecolor{gray}{rgb}{0.5,0.5,0.5}
\definecolor{mauve}{rgb}{0.58,0,0.82}
\newcommand{\HRule}{\rule{\linewidth}{0.5mm}}
\newcommand{\siatkonator}{\textbf{Siatkonator} }
\renewcommand{\triangle}{\href{http://www.cs.cmu.edu/~quake/triangle.html}{triangle}}
\lstset{ %
  basicstyle=\ttfamily\footnotesize,
  numbers=left,
  numberstyle=\footnotesize,
  stepnumber=1,
  numbersep=5pt,
  breaklines=true,
  tabsize=2,
  showspaces=false,
  showstringspaces=false,
  frame=single,
  numberstyle=\tiny\color{gray},
  keywordstyle=\color{mauve},
  commentstyle=\color{dkgreen},
  stringstyle=\color{mauve},
}

\begin{document}
\begin{titlepage}

  \begin{center}


    % Upper part of the page
    \includegraphics[width=0.3\textwidth]{logo.jpg}\\[1cm]

    \begin{onehalfspace}
      \textsc{\LARGE Wydział Elektryczny Politechniki Warszawskiej}\\[1.5cm]
    \end{onehalfspace}



    \textsc{\Large Języki I~Metody Programowania II}\\[0.5cm]

    % Title
    \HRule \\[0.4cm]
    {\huge \bfseries Siatkonator }\\[0.2cm]
    \HRule \\[1.5cm]

    % Author and supervisor
    \begin{flushleft} \large
      \emph{Autor:}\\
      Barnaba \textsc{Turek}
    \end{flushleft}
    \vfill

    % Bottom of the page
    {\large \today}

  \end{center}

\end{titlepage}
\sloppy

\setcounter{tocdepth}{4}
\tableofcontents

\section{Specyfikacja funkcjonalna}
\subsection{Program}
\siatkonator to program sklejający siatki trójkątne.
Program przyjmuje jedną lub więcej siatek trójkątnych oraz jeden wielokąt.

Wynikiem działania programu jest siatka trójkątna opisująca zadany wielokąt, która zawiera podane określone siatki.

\begin{figure}[h]
  \centering
  \includegraphics[width=0.5\textwidth]{ilustracja.png}
  \caption{przykład sklejania zadanego wielokątu W~i~siatek A~i~B}
\end{figure}

\subsection{Pliki}
\siatkonator korzysta z~plików w~formatach \texttt{poly}, \texttt{ele} i~\texttt{node}.

\begin{description}
  \item[poly] \href{http://www.cs.cmu.edu/~quake/triangle.poly.html}{format pliku opisującego wielokąt}
  \item[ele] \href{http://www.cs.cmu.edu/~quake/triangle.ele.html}{format pliku opisującego z~których wierzchołków składają się trójkąty siatki}
  \item[poly] \href{http://www.cs.cmu.edu/~quake/triangle.poly.html}{format pliku opisującego wierzchołki siatki}
\end{description}

Wszystkie wykorzystywane formaty są zgodne z~formatami wykorzystywanymi przez program \triangle.

\subsection{Wywołanie}
Program nie prowadzi dialogu z~użytkownikiem.

\begin{lstlisting}[caption=Przykładowe wywołanie]
  $ siatkonator -e siatka1.ele -e siatka2.ele polygon.poly
\end{lstlisting}

Ostatni argument programu określa nazwę pliku (wraz ze ścieżką) w~którym opisany jest wielokąt.
Następne argumenty określają opcje wykonania programu i~nie są wymagane.

Dostępne są następujące opcje:
\begin{description}
  \item[\texttt{-e <name>}] dodaje siatkę, która zostanie ``sklejona'' z~wielokątem. Program zakłada, że istnieje także plik o~takiej samej nazwie bazowej z~rozszerzeniem \texttt{node} opisujący wierzchołki.
  \item[\texttt{-o <name>}] podaje nazwę pliku wyjściowego. W~przypadku braku tego parametru wynik zostanie wypisany na standardowe wejście (najpierw plik ele, potem node).
  \item[\texttt{-a <area>}] określa maksymalną powierzchnię trójkąta (nie zmienia zadanych siatek!).
  \item[\texttt{-q <degrees>}] określa minimalną miarę kąta w~trójkącie w~wygenerowanej siatce trójkątnej (nie zmienia zadanych siatek!),
\end{description}

W~przypadku powodzenia program zwraca zero. W~przeciwnym wypadku program zwraca jeden z~kodów błędu:

\begin{description}
  \item[\texttt{kod 1}] Błąd otwarcia pliku (spowodowany np. brakiem uprawnień lub pliku).
  \item[\texttt{kod 2}] Błąd wczytania pliku (spowodowany błędnym formatem któregoś z~plików źródłowych).
  \item[\texttt{kod 3}] Błędny format argumentów linii poleceń.
\end{description}

Ponadto program w~czasie działania wypisuje informacje o~swoim aktualnym stanie na wyjście \texttt{STDERR}.

\subsection{Dodatkowe uwagi}
Odradzam użytkownikowi podawanie programowi siatek, które przecinają się ze sobą, bądź przecinają się z~podanym wielokątem.
Jeżeli użytkownik koniecznie chce podać bezsensowne dane, to otrzyma bezsensowne wyniki.
Taka sytuacja nie jest rozumiana jako błąd programu i~nie jest do niej przypisany żaden kod błędu.

\section{Specyfikacja implementacyjna}

\end{document}
